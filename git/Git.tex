\documentclass[11pt,twoside,headings=normal,open=right,french,DIV=12]{scrreprt}


% =========================================
% CONFIGURATION
% =========================================


% -------------------------------
% Configuration langue/caractères
% -------------------------------
\usepackage[utf8]{inputenc}
\usepackage[T1]{fontenc}
\usepackage{babel}


% -------------------------------
% Packages
% -------------------------------
\usepackage{lmodern}                         % fonte de caractères
\usepackage{graphicx}                        % insertion d'images
\usepackage{hyperref}                        % gestion des hyperliens					
\hypersetup{                                 % paramètres hyperref
	colorlinks=true,           						% activer la couleur pour les liens	             
    linkcolor=blue} 								% couleur bleue pour liens internes
\usepackage[french]{cleveref}                % références croisées



% ------------------
% Gestion des images
% ------------------
\graphicspath{{./images/}}
\DeclareGraphicsExtensions{.png}


% -------------------------------
% Les options LaTeX
% -------------------------------
\pagestyle{headings}                            % style de page


% -------------------------------
% Page de titre
% -------------------------------
\title{\rule{\linewidth}{1pt}\\ \Huge GIT et GITHUB}
\subtitle{\vspace{\fill}\huge Notes\\[-3mm]\rule{\linewidth}{1pt}}
\publishers{le 28 septembre 2020}
\author{Jérôme JONDEAU}
\date{}


% -------------------------------
% Les options KOMA-Script
% -------------------------------
% Formatage du nom de chapitre
\KOMAoptions{chapterprefix=true,DIV=last}


% -------------------------------
% Commande
% -------------------------------
    % ---------------------------------------------------------------------------------
    % BLOC TEXTE
    % ==========
    %
    %    Objectif    - Insérer le text "Git" en petites capitales
    %    Paramètre   - Nombre   - 0
    % ---------------------------------------------------------------------------------
    \newcommand{\git}{\textsc{Git}}

    % ---------------------------------------------------------------------------------
    % BLOC TEXTE
    % ==========
    %
    %    Objectif    - Insérer le text "GitHub" en petites capitales
    %    Paramètre   - Nombre   - 0
    % ---------------------------------------------------------------------------------
    \newcommand{\github}{\textsc{GitHub}}
    
    % ---------------------------------------------------------------------------------
    % BLOC FORMATAGE TEXTE
    % ====================
    %
    %    Objectif    - Formater un texte en typewriter
    %    Paramètre   - Nombre   - 1
    %                  Fonction - Le texte à formater    
    % ---------------------------------------------------------------------------------
    \newcommand{\spec}[1]{\texttt{#1}}

    % ---------------------------------------------------------------------------------
    % BLOC FORMATAGE TEXTE
    % ====================
    %	
    %    Objectif    - Formater un texte en "confère" (cf)
    %    Paramètre   - Nombre   - 1
    %                  Fonction - Le texte à formater
    % ---------------------------------------------------------------------------------
    \newcommand{\cf}[1]{{\footnotesize{cf #1}}}
    
    % ---------------------------------------------------------------------------------
    % BLOC IMAGE
    % ==========
    %
    %    Objectif    - Insérer une image
    %    Paramètre   - Nombre   - 2
    %    Fonction - #1 : la valeur d'échelle
    %                             #2 : l'intitulé du fichier image à utiliser
    % ---------------------------------------------------------------------------------      
    \newcommand{\img}[2]{\includegraphics[scale=#1]{#2}} 
    
    % ---------------------------------------------------------------------------------
    % BLOC IMAGE
    % ==========
    %
    %    Objectif    - Insérer une image dans une boîte, elle même dans le flus de texte
    %    Paramètre   - Nombre   - 3
    %                  Fonction - #1 : le décalage vertical de la boîte (par rapport à la ligne de base)
    %                             #2 : le facteur d'échelle
    %                             #3 : 'intitulé du fichier image à utiliser
    % ---------------------------------------------------------------------------------      
    \newcommand{\rimg}[3]{\raisebox{#1}{\includegraphics[scale=#2]{#3}}}

    % ---------------------------------------------------------------------------------
    % BLOC INFORMATION
    % ================
    %
    %    Objectif    - Insérer un bloc de commentaire dans le flux du texte.
    %    Paramètre   - Nombre   - 1
    %                  Fonction - #1 : le texte de l'information
    % ---------------------------------------------------------------------------------
	\newcommand{\info}[1]{
    \begin{minipage}{\textwidth}
	    % Icône
	    % ------
	    \begin{minipage}[c]{0.1\linewidth}
		    \centering\img{0.1}{../images/info}
        \end{minipage}    
	    % Contenu
	    % ---------
		\begin{minipage}[c]{0.8\linewidth}
		    #1
        \end{minipage}
    \end{minipage}
	}



% =========================================
% LE DOCUMENT
% =========================================
\begin{document}



    % -------------------------------
    % (Re)Définition
    % -------------------------------
    % Titre de la table des matières
    \renewcommand{\contentsname}{Sommaire}



	% Page de titre
	% -------------
	\maketitle


    
    % Sommaire
    % ------------
    \tableofcontents



\addchap{Informations}



	Les informations typographiées avec une police de \texttt{type machine à écrire} représentent des noms ou des entrées de menu.
	
	Dans un lecteur de fichier \emph{pdf} les parties de texte présentées en caractères bleus sont des liens
	hypertextes pointant vers un partie du document.
	


\chapter{Préparation et démarrage}	



\section{Informations}



    Cette section présente l'application \git{} et indique comment préparer son utilisation.


\subsection{Présentation}



    \git{} est un système de gestion de versions, de l'anglais \textbf{V}ersion \textbf{C}ontrol \textbf{S}ystem,
    communément raccourci en \textsc{VCS}. Selon leur principe de fonctionnement, les \textsc{VCS} se répartissent
    en trois systèmes principaux :
    \begin{enumerate}
        \item local ou \textsc{VCS},
        \item centralisé ou \textbf{C}\textsc{entralized VCS},
        \item distribué ou \textbf{D}\textsc{istributed VCS}.
    \end{enumerate}

    \smallskip

    \git{} est un \textsc{VCS}. Contrairement à d'autres solutions, il n'est pas prévu pour suivre un flux de
    modifications mais plutôt un flux d'instantanés, où chaque instantané peut être considéré comme \og{}un mini système de
    fichiers\fg. Les opérations sont donc locales, chaque validation est signée en \textsc{SHA-1} et nommée par sa signature.

    \smallskip

    Il n'existe que deux statuts de données : \textsc{non suivi} et \textsc{suivi}. Au cours de son \textsc{suivi} de version,
    une donnée aura l'un des trois états suivants :
    \begin{description}
        \item \textsc{modifié} le contenu a été modifié, il n'a pas encore été ni indexé ni validé ;
        \item \textsc{indexé :} le contenu \textsc{modifié} a été pris en compte et est en attente de validation ;
        \item \textsc{validé: } le contenu \textsc{indexé} a été enregistré dans la base de données. Il servira de référence pour le
        prochain instantané.
    \end{description}
    
    En correspondance, ces trois états reflètent trois zones de \git{} :
    \begin{description}
        \item {le répertoire de travail :} la zone de modification des données ;
        \item {la zone d'index : } les données qui seront utilisées pour le prochain instantané ;
        \item {le répertoire .git : } l'ensemble des méta-données et la base de données.
    \end{description}
    
    Le mode opératoire global de suivi est donc lui aussi logiquement réparti en trois étapes :
    \begin{enumerate}
        \item modification des données,
        \item sélection et indexation des données modifiées,
        \item validation des données indexées.
    \end{enumerate}        



\subsection{CLI et GUI}


    
    \git{} est utilisable en ligne de commande et au travers d'interfaces graphiques. Bien qu'utiliser la ligne de
    commande puisse paraître d'un autre âge, elle offre deux avantages importants :
    \begin{enumerate}
        \item les commandes sont identiques sur toutes les plateformes,
        \item toutes les commandes sont disponibles.
    \end{enumerate}
    
    En comparaison, les interfaces graphiques proposent un aspect et un fonctionnement spécifiques à chaque application.
    Mais leur véritable inconvénient est de ne proposer qu'un sous-ensemble -- souvent restreint -- des commandes
    disponibles.



\section{Configuration initiale}



La version \git{} est obtenue, sous \bsc{Git bash} et \bsc{Git cmd}, par la commande \verb|git --version|.


    Cette section présente la configuration de \git{} pour \textsc{Linux} et indique la liste des fichiers de configuration
    pour \textsc{Microsoft Windows}\footnote{A partir de \textsc{Microsoft Windows 7}} .



\subsection{Les fichiers de configuration sous Linux}



    Le fichier \spec{/etc/gitconfig} définit la configuration globale de \git, destinée à tous les utilisateurs.
    La commande \verb|git config --system --list| affiche les paramètres de ce fichier.
    
    Le fichier \spec{\char`\~/.gitconfig} définit la configuration spécifique d'un utilisateur. La commande
    \verb|git config --global --list| affiche les paramètres de ce fichier.
    
    Enfin, un fichier \spec{.git/config} est présent dans chaque dossier de dépôt pour en définir la configuration
    et la structure.



\subsection{Les fichiers de configuration sous Microsoft Windows}



    Le fichier \spec{\%windir\%:\textbackslash{}ProgramData\textbackslash{}Git\textbackslash{}config} définit la configuration
    globale de \git, destinée à tous les utilisateurs.
    
    Le fichier \spec{\%homepath\%:\textbackslash.gitconfig} définit la configuration spécifique d'un utilisateur.
    
    Enfin, le fichier \spec{.git\textbackslash config} est présent dans chaque dossier de dépôt pour en définir la configuration
    et la structure.    



\subsection{Configuration initiale de l'utilisateur}



    Pour définirl'identité de l'utilisateur, deux commandes :
    \begin{enumerate}
        \item \verb|git config user.name "John Doe"|
        \item \verb|git config user.email "john.doe@example.com"|
    \end{enumerate}

    Par défaut sous \textsc{Linux}, \git{} utilise l'éditeur de fichier \textsc{VIM}. La commande suivante permet de
    définir l'éditeur \textsc{emacs} : \verb|git config --global core.editor emacs|.



\subsection{Affichage des paramètres}

    
    
    L'affichage de l'ensemble des paramètres -- global et utilisateur -- est obtenu avec la commande
    \verb|git config --list|. L'éventuel double affichage de certains paramètres est dû au fait qu'ils sont successivement
    lus dans les deux fichiers de configuration : dans ce cas la dernière valeur affichée prévaut.

    \smallskip
    
    L'affichage d'un paramètre est obtenu par la commande \verb|git config <parametre>|. Par exemple, le nom de
    l'utilisateur est affiché avec la commande \verb|git config user.name|.



\section{Obtenir de l'aide}



    Sous Linux l'aide pour une commande est accessible de trois façons :
    \begin{enumerate}
        \item \verb|git help <commande>|
        \item \verb|git <commande> --help|
        \item \verb|man git-<commande>|
    \end{enumerate}



\chapter{Démarre ou cloner un dépôt}



\section{Démarrer un dépôt Git}



\subsection{Depuis un dossier existant}
\label{sec-dossier_existant}


    Le dossier courant est celui dont on se propose de suivre les données (il est ici supposé que ce dossier esdt vide). La commande initiale \verb|git init|
    créé le sous-dossier \verb|.git| qui contiendra les méta-données et la base de données du dépôt. Cette
    commande ne doit être exécutée qu'une seule fois, au moment où \git{} prend en compte le dossier.
    
    \smallskip
    
    La commande \verb|git status| récapitule l'état de suivi des données. Ici l'état indique "dossier propre"
    car le dossier est vide.
    
    \smallskip
    
     Créons le fichier  \spec{README} : la commande \verb|git status| indique que fichier \spec{README} n'est pas suivi. La commande \verb|git add README| indexe ce fichier : la commande \verb|git status| indique maintenant qu'il est suivi.
     
    \smallskip    
    
    La commande \verb|git commit -m "commit initial"| valide le fichier indexé pour créer un instantané, référencé par une signature numérique SHA-1. La commande \verb|git status| indique de nouveau "dossier propre".



\subsection{Cloner un dépôt}



    Le clonage va rapatrier le dépôt distant dans le répertoire courant en y créant un dossier ad-hoc. 
    
    \smallskip
    
    Admettons que le dépôt à cloner soit \spec{https://github.com/libgit2/libgit2}. La commande
    \verb|git clone https://github.com/libgit2/libgit2| procède aux opérartions suivantes :
    \begin{enumerate}
        \item création locale du dossier \spec{libgit2},
        \item récupération de la base de données du dépôt,
        \item initialisation du sous-dossier .git,
        \item extraction de la dernière version.    
    \end{enumerate}

    Pour personnaliser le dossier local en \spec{monlibgit2}, la commande devra être modifiée pour spécifier le dossier,
    soit \verb|git clone https://github.com/libgit2/libgit2 monlibgit2|.
    
    \smallskip
    
    Dès lors, le suivi de version est identique à celui présenté à la section \ref{sec-dossier_existant}.



\chapter{Les commandes de base}



\section{Le statut court}



    La commande \verb|git status -s| affiche un état synthétique du suivi de données, sur deux colonnes : la colonne
    de gauche représente l'état de l'index, celle de droite celui du répertoire de travail. Ci-dessous les différents états de suivi :
    \begin{description}
	    \item {"\_M"} : fichier indexé modifié mais pas encore indexé
        \item {"MM"} : fichier modifié et indexé puis de nouveau modifié
		\item {"M\_"} : fichier modifié et indexé
		\item {"AM"} : nouveau fichier indexé puis modifié        
		\item {"A\_"} : nouveau fichier indexé
	    \item {"??"} : fichier non suivi
	\end{description}    



\section{Indexer des fichiers}



    Les commandes suivantes présentes des exemples courants d'indexation :
    \begin{description}
        \item {\verb|git add readme|} : indexer le fichier \spec{readme}
        \item {\verb|git add *.c|} : indexer tous les fichier d'extension \spec{c}
        \item {\verb|git add nom_dossier|} : dans le cas d'un dossier, tous les éléments du dossier sont indexés, en mode récursif
    \end{description}



\section{Ignorer des fichiers}



\section{Fichiers d'exclusion}



Un fichier d'exclusion, à nommer \spec{.gitmore}, liste les modèles de nom de fichiers ou dossiers à exclure
du suivi. Un fichier d'exclusion s'applique récursivement sur toute la structure du dossier de travail ou jusqu'à rencontrer un nouveau fichier d'exclusion.



\subsection{Définir les exclusions}



    Chaque ligne du fichier \spec{.gitmore} définit une exclusion, comme indiqué ci-dessous.
    \begin{itemize}
    
        \item une ligne vide est ignorée
        \item une ligne de commentaire, qui doit débuter par \#, est ignorée        
        \item ensemble de caractères :
        \begin{itemize}
        		\item pour les noms de fichiers ou de dossiers
        		\item \spec{[bB]}uild exclut \spec{build} et spec{Build}
        \end{itemize}
        \item patron de fichier :
		\begin{itemize}
			\item \spec{dir1/dir2/file.ext} :
			\begin{itemize}
				\item exclut le fichier {file.ext} en récursif
				\item à partir du dossier {dossier de travail/dir1/dir2/}
			\end{itemize}
			\item \spec{otherfile.ext} :
			\begin{itemize}
				\item exclut le fichier {otherfile} en récursif
				\item à partir du dossier {dossier de travail}
			\end{itemize}
			\item \spec{doc/*.apk} : 
			\begin{itemize}
				\item exclut tous fichiers d'extension \spec{apk}
				\item à partir du dossier \spec{doc} en mode résursif
			\end{itemize}		
			\item \spec{/*.apk} : 
			\begin{itemize}
				\item exclut tous fichiers d'extension \spec{apk}
				\item uniquement dans le dossier courant (sans récursion)
			\end{itemize}					
			\item exception à l'exclusion : 
			\begin{itemize}
				\item\spec{!*}
				\item \spec{!a.apk}
				\item exclut tous les fichiers d'extension \spec{apk} sauf \spec{a.apk}
				\item uniquement dans le dossier courant (sans récursion)
			\end{itemize}				
		\end{itemize}			        
        \item patron de dossier :
		\begin{itemize}        
			\item \spec{dir1/} : exclut le dossier {dir1} lui-même et en récursif
			\item il faut spécifier un dossier par lig
		\end{itemize}		
        \item : \spec{**/dir1/} :
		\begin{itemize}        
			\item exclut le dossier \spec{dir1} en mode récursif
			\item sans exclure les fichiers \spec{dir1}
		\end{itemize}		
        \item : \spec{dir1/**/dir2} :
		\begin{itemize}        
			\item exclut le dossier \spec{dir2}
			\item quelqu soit le nombre de dossiers entre \spec{dir1} et \spec{dir2}
		\end{itemize}		
        \item patron commun fichier et dossier :
		\begin{itemize}        
			\item \spec{elem} : exclut tous les fichiers et tous dossiers en mode résursif			
		\end{itemize}							
    \end{itemize}



\subsection{Exemples}



    \begin{description}       
	    \item \# pas de fichier .a
		\item \spec{.a}
        \item \# conserver lib.a malgré la règle précédente
		\item \spec{!lib.a}
        \item \# ignorer le fichier TODO à la racine du projet
		\item \spec{/TODO}
	    \item \# ignorer tous les fichiers dans le dossier build
		\item \spec{build/}
		\item \#ignorer doc/note.txt mais pas doc/server/arch/txt
		\item \spec{doc/*.txt}
    \end{description}

	
	
	
\section{Inspecter les modifications non indexées et indexées}



\subsection{Les modifications non indexées}



    La commande \verb|git diff| compare le contenu du dossier de travail avec la zone d'index pour établir la liste toutes les modifications qui n'ont pas encore été indexées.
    
    \smallskip
    
    Si toutes les modifications ont été indexées, la liste est vide. Sinon elle affiche les fichiers modifiés et les modifications apportées à chacun d'entre eux. Il est possible de visualiser cette liste sous forme graphique avec la commande \verb|git difftool|.
    
La commande \verb|git difftool --tool-help|
    liste à son tour les applications disponibles pour cet affichage.



\subsection{Les modifications indexées}



    La commande \verb|git diff --staged| compare la zone d'index avec le dernier instantané pour établir la liste de toutes
    les modifications indexées. Ces modifications seront celles prises en compte lors de la prochaine validation.



\section{Valider les modifications}



    Deux méthodes permettent de valider les données indexées.



\subsection{Via la commande d'édition}    
\label{subsec-valider_commit}


    La commande \verb|git commit [-v]| lance l'éditeur défini dans le configuration \git{} pour permettre la saisie de l'intitulé de commit.
    
    Sans le paramètre \verb|-v| la liste des nom de fichiers à valider est afficher. Avec le paramètre un \verb|git diff| de chaque fichier indexé est affiché.
    
    \smallskip
    
    Une fois l'intitulé du commit saisi sur la première ligne, le fichier édité doit être fermé avec sauvegarde. Le
    \verb|commit| est alors automatiquement réalisé avec l'intitulé saisi.



\subsection{Via la commande directe}    



    La commande \verb|commit -m "intitulé"| réalise directement le commit en l'intitulant avec le texte indiqué en dernier paramètre.

    \smallskip
    
    La commande \verb|commit -am "intitulé"| indexe automatiquement tous les fichiers déjà en suivi de version et produit directement le commit avec le texte indiqué en dernier paramètre.



\section{Supprimer ou renommer un fichier}



\subsection{Supprimer un fichier}



    Deux cas de figure sont envisageables : demander à \git{} de supprimer le fichier ou le supprimer directement de dossier de travail.



\subsubsection{Suppression par Git}    



    \git{} supprime le fichier du dossier de travail, les éventuelles informations de suivi puis indexe automatiquement la suppression.
    
    Si ce fichier n'est pas indexé, la commande \verb|git rm nomFichier| procèdera à la
    suppression. S'il est indexé, il faudra forcer la suppression avec la commande
    \verb|git rm -f nomFichier|.



\subsubsection{Suppression directe du fichier}    



    La suppression directe d'un fichier dans le dossier de travail sera indiquée par la commande \verb|git status|. Il faudra alors appeler \verb|git rm [-f] nomFichier| pour indexer la suppression.



\subsubsection{Un cas pratique}    



Comment supprimer le suivi d'un fichier, par exemple suite à une modification \verb|.gitignore| ?
En utilisant la commande \verb|git rm --cached nomFichier|.



\subsection{Renommer un fichier}



    La commande \verb|git mv nom_origine nom_cible| renomme le fichier \spec{nom\_origine} en \spec{nom\_cible}.
    Cette commande est directement indexée. Les trois commandes suivantes seraient nécessaires pour produire le même résultat :
    \begin{enumerate}
        \item \verb|mv nom_origine nom_cible|
  		\item \verb|git rm nom_origine|
		\item \verb|git add  nom_cible|
    \end{enumerate}



\section{Visualiser l'historique des validations}


  
La commande \verb|git log| affiche tous les \spec{commit} réalisés, le dernier apparaîsant en premier. Cet affichage est réalisé par un outil de pagination.
    
    \smallskip
    
    La commande \verb|git log| peut être assortie de très nombreux paramètres. Les plus utilisés sont listés ci-dessous :
    \begin{description}
            \item {-p} : afficher les différences introduites pour chaque validation,
        \item {-x} : limiter l'affichage aux \spec{x} commits les plus récents,
        \item {--stat} : afficher les statistiques résumées de chaque commit,
        \item {-Snom} : limiter l'affichage aux commits qui ajoute ou retire du texte comportant \spec{nom},
        \item {--pretty} : modifier le format des informations de sortie.
    \end{description}



\section{Modification d'un commit}



Le \verb|commit| erroné ne peut pas être modifié : il faut donc le remplacer.


\subsection{Modifier l'intitulé d'un commit}



La commande \verb|git commit --amend| permet de saisir un nouvel intitulé, comme présenté à la section \ref{subsec-valider_commit}.



\subsection{Ajouter un fichier au commit}


    
    Pour ajouter un \spec{fichier} oublié, il suffit de l'indexer avec la commande \verb|git add nomFichier| puis de lancer la commande \verb|git commit --amend| : la saisie de l'intitulé est alors  réalisé comme présenté à la section \ref{subsec-valider_commit}.



\subsection{Désindexer un fichier}



    Il peut être nécessaire de désindexer un fichier dans l'objectif de réaliser un \spec{commit} dédié à la modification qu'il apporte. La commande \verb|git reset HEAD nomFichier| désindexe le fichier \spec{nomFichier}. Il faut alors réindexer le fichier et créer le \spec{commit} dédié.



\subsection{Réinitialisation}



\subsubsection{Réparer une erreur non validée}



%	\info{\textbf{Attention !} L'ancien contenu est \textbf{définitivement} perdu.} 

La commande \verb|git reset --hard HEAD| restaure le dossier de travail et l'index à l'état du dernier \verb|git commit|. Les commandes \verb|git diff| et \verb|git diff --cached| ne rapporteront donc plus aucune différence.

    \smallskip

Pour restaurer un seul fichier, deux commandes sont nécessaires :
\begin{enumerate}
\item \verb|git checkout -- nomFichier| : restauration du fichier à la version de l'index (la commande \verb|git diff| ne rapportera plus de différence),
\item \verb|git checkout HEAD nomFichier| : restauration de l'index à la valeur du dernier \verb|commit| (la commande \verb|git diff --cached| ne rapportera plus de différence).
\end{enumerate}



\subsubsection{Réparer une erreur validée}



Il faut créer un nouveau \verb|commit| qui annule les changements du \verb|commit| erroné. Deux  exemples :
\begin{itemize}
\item la commande \verb|git revert HEAD| annule les changements du dernier \verb|commit (HEAD)| et lance l'édition d'un intitulé,
\item la commande \verb|git revert HEAD^| annule les changements de l'avant dernier \verb|commit (HEAD^)| et lance l'édition d'un intitulé.
\end{itemize}



\section{Connexion avec des dépôts distants}    



\subsection{Afficher les connexions distantes}    



    La commande \verb|git remote| affiche la liste des dépôts distants enregistrés dans le dépôt local. Dans cette liste, le terme \spec{origin} est le nom par défaut que \git{} attribue à un dépôt cloné.
    
    \smallskip
    
    La commande \verb|git remote -v| ajoute l'url des dépôts à la liste affichée.



\subsection{Ajouter une connexion distante sur un nom court}



    En enregistrant un dépôt distant il est possible de lui associer un non court qui pourra être utiliser dans toutes
    les opérations. Dans la liste \verb|git remote|, le dépôt ne sera plus repéré par \verb|origin| mais par son nom court.
    
    \smallskip
    
    La commande \verb|git remote add pb https://github.com/pailboone/ticgit| associe le nom court \verb|pb| au dépôt distant \verb|\https://github.com/pailboone/ticgit|.



\subsection{Inspecter les informations d'une connexion}



    La commande \verb|git remote show nonDistant| affiche un récapitulatif complet des informations de la connexion.

%	\info{\texttt{nonDistant} est le nom local associé au dépôt distant : \texttt{origin} ou nom court.}  



\subsection{Renommer un dépôt distant}



    La commande \verb|git remote rename nonDistant nouveauNNomDistant| modifie le nom court d'une connexion. Les noms des branches sont automatiquement mis en jour.

%	\info{\texttt{nonDistant} est le nom local associé au dépôt distant : \texttt{origin} ou nom court.}     



\subsection{Retirer un dépôt local}


    La commande \verb|git remote rm nonDistant| retire la connexion distante \verb|nonDistant|.


%	\info{\texttt{nonDistant} est le nom local associé au dépôt distant : \texttt{o%rigin} ou nom court.}     
  



\subsection{Récupérer et tirer un dépôt distant}        



    La commande \verb|git fetch nonDistant nomBranche| récupère les données mises à jour sur la branche \verb|nomBranche| du dépôt distant et absentes du dépôt local. Ces données sont copiées dans une branche locale qui n'est pas une branche de travail. La commande \verb|git merge| devra être utilisée pour fusionner ces informations sur une branche locale.
    
    La commande \verb|git pull origin master| opère la même opération mais procède également à la fusion des données récupérées dans la branche \verb|master|. Elle équivant aux deux commandes successives \verb|git fetch| et \verb|git merge|.

%	\info{\texttt{nonDistant} est le nom local associé au dépôt distant : \texttt{origin} ou nom court.} 	



\subsection{Pousser vers un dépôt distant}    



    La commande \verb|git push origin master| envoie les modifications de la branche locale \verb|master| du dépôt distant.

%	\info{\texttt{nonDistant} est le nom local associé au dépôt distant : \texttt{origin}, nom court.}      
    
    \smallskip
    
    Pour que cette commande fonctionne il faut évidemment disposer de droits d'écriture sur le distant. De plus, toutes les modifications du distant doivent avoir été récupérées en local : dans le cas contraire il sera d'abord nécessaire de fusionner ces données en local -- commande \verb|git pull| -- puis de pousser vers le distant.



\section{\'Etiquetage}    



\subsection{Introduction}



    Etiqueter un commit permet de localiser un état précis dans l'historique. La commande \verb|git tag| affiche la liste
    alphabétiques des étiquettes définies. Il est possible de rechercher un motif particulier, par exemple
    \verb|git tag -l 'v1.8.5*'|.



\subsection{Créer des étiquettes}    



    Il existe deux type d'étiquettes : annotée et légère.



\subsubsection{\'Etiquette annotée}



    Une étiquette annotée dispose d'une somme de contrôle, contient le nom et l'adresse de courriel du créateur, la date
    et un message d'étiquetage. Elle peut en outre être signée et vérifiée avec \textsc{Gnupg} : elle est stockée dans
    la base de données \verb|git|.
    
    \smallskip
    
    La commande \verb|git tag -a v1.4| lance l'éditeur défini dans la configuration \git{} pour saisir le message d'étiquetage. La fermeture
    sauvegardée du fichier édité créé l'étiquette. La commande \verb|git tag -a v1.4 -m 'version 1.4'| procède directement
    à la création de l'étiquette. Dans tous les cas le dernier \spec{commit} est étiquetté par \verb|v1.4|.
    
    \smallskip
    
    La commande \verb|git show v1.4| affiche conjointement les informations de l'étiquette et du \spec{commit} associé.



\subsubsection{\'Etiquettes légère}



    Une étiquette légère stocke uniquement la somme de contrôle d'un \spec{commit} dans un fichier annexe. La commande
    \verb|git tag v1.4g| créé l'étiquette légère \verb|v1.4g| sur le dernier \spec{commit}.
    
    \smallskip
    
    La commande \verb|git show v1.4g| affiche uniquement les informations du \spec{commit}.



\subsubsection{\'Etiqueter un commit précis}    



    Il est possible d'étiqueter un \spec{commit} précis, en indiquant tout ou partie de sa somme de contrôle. La commande \verb|git tag -a v1.2 9fceb02| créé une étiquette annotée pour le commit dont la
    somme de contrôle débute par \verb|9fceb02|.



\subsubsection{Partager les étiquettes}



    Par défaut, la commande \verb|git push| ne pousse pas les étiquettes. Pour pousser l'étiquette \verb|v1.2| il faut
    utiliser la commande \verb|git push origin v1.2|. Pour pousser toutes les nouvelles étiquettes en une seule opération il faut utiliser
    la commande \verb|git push origin --tags|.



\section{Les alias}


   
    L'alias \verb|git config --global alias.last 'log -1 HEAD'| affiche le dernier \spec{commit}. L'appel sera
    \verb|git last|.

    \smallskip
    
    L'alias \verb|git config --global alias.ci 'commit'| exécute le \spec{commit}. L'appel sera \verb|git ci|.

    \smallskip
    
    L'alias \verb|git config --global alias.st 'status'| affiche le statut.. L'appel sera \verb|git st|.



\chapter{Cas pratiques : suivi}



\section{Initialisation et suivi Git}



Cette section résume un cas d'école : comment initialiser le suivi \git{} d'un dossier puis comment prendre en compte le suivi de la création et de la modification d'un fichier dans ce dossier.

%	\info{Les étapes et commandes listées sont celles utilisées sous \bsc{Git bash} ou \bsc{Git cmd}.}   



\subsection{Initialisation}



\begin{enumerate}
\item se déplacer dans le dossier
\item \verb|git init| : intialiser le suivi
\item \verb|git status|  : pour démonstration, afficher l'état actuel du suivi 
\end{enumerate}



\subsection{Création et suivi d'un nouveau fichier}



\begin{enumerate}
\item créer un fichier \verb|nomFichier| dans le dossier
\item \verb|git status|  : le nouveau fichier est détecté comme non "suivi"
\item \verb|git add nomFichier| : indexer le fichier (cqfd : prise en compte de son suivi)
\item \verb|git commit -m "Premier fichier"| : valider la version du nouveau fichier
\item git log : afficher les informations du commit 
\item \verb|git status|  : il n'y a plus de fichier à suivre
\end{enumerate}



\subsection{Modifier er suivre le fichier}



\begin{enumerate}
\item modifier le fichier \verb|nonFichier|
\item \verb|git status|  : la modification du fichier est détectée comme non "suivie"
\item \verb|git add nomFichier| : indexer le fichier (cqfd : prise en compte de sa modification)
\item \verb|git commit -m "Modification premier fichier"| : valider la version modifiée du fichier
\item \verb|git log| : afficher les informations des commits 
\item \verb|git status|  : il n'y a plus de modification à suivre
\end{enumerate}    
end{itemize}



\subsection{Commandes push et pull}



\subsubsection{Commande pull}



\begin{enumerate}
\item créer un dépôt GitHub
\item \verb|git pull urlDepotGitHub| (url indiquée dans les paramètres GitHub du dépôt)
	\begin{enumerate}
	\item \git{} contacte le dépôt distant, sur son url
	\item il créé un sous-dossier du nom du dépôt et s'y déplace
	\item il télécharge la base de données
	\item il extrait la base de données
	\end{enumerate}
\end{enumerate}    



\subsubsection{Commande push}



\begin{enumerate}
\item créer un dépôt GitHub
\item se déplacer dans le dossier du dépôt local
\item \verb|git remote add origin urlDepotGitHub| (url indiquée dans les paramètres GitHub du dépôt)
\item \verb|git push origin master|
\end{enumerate}  



 
 
\end{document}

